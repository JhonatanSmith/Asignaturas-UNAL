% Options for packages loaded elsewhere
\PassOptionsToPackage{unicode}{hyperref}
\PassOptionsToPackage{hyphens}{url}
%
\documentclass[
]{article}
\usepackage{lmodern}
\usepackage{amsmath}
\usepackage{ifxetex,ifluatex}
\ifnum 0\ifxetex 1\fi\ifluatex 1\fi=0 % if pdftex
  \usepackage[T1]{fontenc}
  \usepackage[utf8]{inputenc}
  \usepackage{textcomp} % provide euro and other symbols
  \usepackage{amssymb}
\else % if luatex or xetex
  \usepackage{unicode-math}
  \defaultfontfeatures{Scale=MatchLowercase}
  \defaultfontfeatures[\rmfamily]{Ligatures=TeX,Scale=1}
\fi
% Use upquote if available, for straight quotes in verbatim environments
\IfFileExists{upquote.sty}{\usepackage{upquote}}{}
\IfFileExists{microtype.sty}{% use microtype if available
  \usepackage[]{microtype}
  \UseMicrotypeSet[protrusion]{basicmath} % disable protrusion for tt fonts
}{}
\makeatletter
\@ifundefined{KOMAClassName}{% if non-KOMA class
  \IfFileExists{parskip.sty}{%
    \usepackage{parskip}
  }{% else
    \setlength{\parindent}{0pt}
    \setlength{\parskip}{6pt plus 2pt minus 1pt}}
}{% if KOMA class
  \KOMAoptions{parskip=half}}
\makeatother
\usepackage{xcolor}
\IfFileExists{xurl.sty}{\usepackage{xurl}}{} % add URL line breaks if available
\IfFileExists{bookmark.sty}{\usepackage{bookmark}}{\usepackage{hyperref}}
\hypersetup{
  hidelinks,
  pdfcreator={LaTeX via pandoc}}
\urlstyle{same} % disable monospaced font for URLs
\usepackage[margin=1in]{geometry}
\usepackage{longtable,booktabs}
\usepackage{calc} % for calculating minipage widths
% Correct order of tables after \paragraph or \subparagraph
\usepackage{etoolbox}
\makeatletter
\patchcmd\longtable{\par}{\if@noskipsec\mbox{}\fi\par}{}{}
\makeatother
% Allow footnotes in longtable head/foot
\IfFileExists{footnotehyper.sty}{\usepackage{footnotehyper}}{\usepackage{footnote}}
\makesavenoteenv{longtable}
\usepackage{graphicx}
\makeatletter
\def\maxwidth{\ifdim\Gin@nat@width>\linewidth\linewidth\else\Gin@nat@width\fi}
\def\maxheight{\ifdim\Gin@nat@height>\textheight\textheight\else\Gin@nat@height\fi}
\makeatother
% Scale images if necessary, so that they will not overflow the page
% margins by default, and it is still possible to overwrite the defaults
% using explicit options in \includegraphics[width, height, ...]{}
\setkeys{Gin}{width=\maxwidth,height=\maxheight,keepaspectratio}
% Set default figure placement to htbp
\makeatletter
\def\fps@figure{htbp}
\makeatother
\setlength{\emergencystretch}{3em} % prevent overfull lines
\providecommand{\tightlist}{%
  \setlength{\itemsep}{0pt}\setlength{\parskip}{0pt}}
\setcounter{secnumdepth}{-\maxdimen} % remove section numbering
\ifluatex
  \usepackage{selnolig}  % disable illegal ligatures
\fi

\author{}
\date{\vspace{-2.5em}}

\begin{document}

\(\rule{6.5in}{1pt}\)

\begin{center}

\textbf{UNIVERSIDAD NACIONAL DE COLOMBIA}

\textit{TRABAJO FINAL DE MUESTREO}

\textbf{INTEGRANTES:}

\textit{DANIELA PICO}

\textit{JHONATAN SMITH}

\textit{KALINE RIOS}

\textit{MARIA FERNANDA YEPES}

\textbf{ESCUELA DE ESTADÍSTICA}

\textbf{2021-01}
\end{center}

\(\rule{6.5in}{1pt}\)

\begin{center}

\textbf{¿Cómo ha afectado la virtualidad al rendimiento académico de los estudiantes en el COLEGIO CAMPESTRE MONTAIGNE?}
\end{center}

El presente informe data los resultados obtenidos al muestrear la
institución educativa COLEGIO CAMPESTRE MONTAIGNE. Dicho colegio es una
entidad privada ubicada al norte del municipio de bello Antioquia (Cl
52, Bello, Medellín, Antioquia). Para la toma de muestras y estimación
de los parámetros de interés, se identifica lo siguiente:

\begin{itemize}
\item
  Es de carácter privado y se ubica en las afueras de la ciudad.
\item
  El colegio oferta sus servicios a los grados 6,7,8,9,10 y 11.
\item
  Tiene un alto componente en idiomas extranjeras.
\item
  La cantidad de alumnos por salón es de 20, aproximadamente. * Con
  algunas excepciones atípicas, generalmente siempre se posee el numero
  de estudiantes por salón ya mencionado.
\item
  Se posee listado de estudiantes con su información académica.
\end{itemize}

En el listado original se observa información básica de cada estudiante.
Identificación, nombre del padre y madre, acudiente, promedio acumulado
y demás datos que podrían no ser relevantes para este problema así que
se filtran los datos para realizar estimaciones de interés.

De esta base de datos NO se posee información de la nota final de cada
una de las asignaturas; solo el promedio final del estudiante al
finalizar el año escolar en un listado general se encuentra organizado
por grados.

Se dice que un estudiante aprueba el año escolar y es promovido al
siguiente grado si su promedio académico acumulado al finalizar el año
es superior a tres (3) en una escala evaluativa de 1 a 5.

NOTA: Se selecciono esta institución educativa por la facilidad de
acceso a los datos pues fueron suministrados en un momento vital de la
pandemia por las directrices vía correo electrónico.

Para la recolecta de los datos e información necesaria para llevar a
cabo el estudio, las directrices de diversas instituciones tanto
publicas como privadas mostraron una posición negativa respecto a la
toma de muestras y la realización de este tipo de trabajos; argumentando
que por motivos de pandemia y privacidad de sus alumnos; se prefiere no
compartir ni brindar información que pudiese llegar a ``comprometer'' la
integridad de los alumnos. Finalmente, la institución privada Campestre
MONTAIGNE accede a la realización del estudio de muestreo con una
condición: solo se tomarán los datos de los estudiantes pertenecientes a
los grados a partir de octavo (es decir, grado 8,9,10,11). De años
anteriores no se tiene información (sexto y séptimo). Cabe destacar que
el colegio no brinda información del por que, pero se sabe qué; el
tamaño total de cada grupo es de 20 estudiantes, además, se sabe que la
institución educativa tiene disponibilidad para dictar clases a partir
de básica secundaria (de sexto a once). Así pues, se asume que los datos
del total de estudiantes en la institución son de 120 sin embargo; al no
disponer de una parte de la muestra y por como se plantea el problema;
podría haber un sesgo en la población.Se decide tomar como ``numero
total de estudiantes matriculados en un año académico en el COLEGIO
CAMPESTRE MONTAIGNE'' un valor de 80.

\begin{center}
\textbf{Diseño Del Muestreo}
\end{center}

Dentro del marco del muestreo se tenia pensado inicialmente realizar el
MAE de dos fases.

En una primera fase del muestreo se identificaban claramente los
estratos y de ellos se realizaba una MAS para cada estrado, de esta
muestra aleatoria de estudiantes se aplicaria una encuesta donde en ella
se obtuviera informacion mas enfocada que pudiese responder las
preguntas problemas.

A continuacion se bosqueja la posible entrevista a llevar a cabo:

Encuestador: Buenas tardes X, a continuacion le realizaré unas breves
preguntas acerca del rendmiento academico que ha llevado en la
institucion educativa donde actualmente cursa el grado Y (por ejemplo,
noveno)

Encuestado: Perfecto.

Encuestador: Muy bien. La primer pregunta es ¿cuantas asignaturas ha
perdido en lo que lleva cursado del año escolar?

Encuestado: Hasta el momento ninguna

Encuestador: Ok, de estas, ¿sabe cuales son las notas para las 12
asignaturas?

Encuestado: Ah, en ingles 4, en frances 4.5, en matematicas, 3.9\ldots{}
etc (En este punto, el estudiante suministraria la nota por asignatura)

Encuestador: Ok. ¿ Cuantas horas a la semana estudia usted en promedio?

Encuestado: Ah no, asi fuera de clase, yo estudio unas 3 horitas

Encuestador: Muy bien. ¿ Cual es la asignatura en la que considera
usted, su rendimiento es mas bajo?

Encuestado: Hmm\ldots{} Yo diria que matematicas.

Encuestador: Finalmente, ¿ha perdido alguna vez una asignatura en esta
institucion?

Encuestado: No, nunca.

Encuestador: Ok, muchas gracias por su tiempo X, ha sido muy amable.

Encuestado: A usted, hasta luego.

Con esta informacion podrian responderse varias preguntas de interes que
no se encuentran dentro del marco teorico o problema inicial propesto
para el trabajo, pero que; debido a la naturalza del muestreo, se
podrian hacer inferencias con dichos datos.

Preguntas como, ¿ cual ha sido la asignatura con el promedio mas alto?
y, ¿con el mas bajo?. ¿Cuanto estudia en promedio un estudiante por
semana fuera de clase?, ¿ Influye esto en su desempeño final?, ¿ Cuantos
estudiantes ha perdido o ganado asignaturas? ¿CUal es esta proporcion?,
¿ Cual es la asignatura con peor percepcion en cuanto a su rendimiento
por parte del estudiante?, ¿ Es esto ultimo cierto?

Respecto a esto ultimo, perder o no una asignatura afecta el promedio
academico obtenido; como es esperable. Sin embargo,

Dentro de todas las preguntas posibles y el marco teorico dado, se
podria generar nuevas variables, nuevas preguntas problema para
modificar la encuesta de tal manera que esta brindase la mayor cantidad
de estudiantes posibles.

Lamentablemente esto no ha sido posible debido a que los colegios han
dado negativas a que los integrantes del equipo de trabajo interactuen
de manera directa con sus alumnos (no se otorga ningun tipo de numero de
contacto y demas); sumado al hecho de que se ha otorgado una base de
datos reducida que se limita aportar la nota promedio obtenida por el
estudiante al finalizar el año escolar.

Se desconoce el motivo del por que no solo esta institucion educativa
hizo una negativa ante compartir y delimitar la interaccion con los
estudiantes. Se asume que una posibilidad (quizas la mas factible) es el
alto estrato del colegio y que su estudiantado podria ser parte de
alguien cercano a una figura publica y demas. El colegio MONTAINGE
recalca que antepone la privacidad de sus estudiantes ante cualquier
otro escenario y menciona que sin importar que; no permitirá el acceso a
estudiantes.

Lo curioso es que; este no fue el unico colegio que se negó a permitir
dicha interaccion. Entre los colegios posibles que se tuvieron en
cuenta, como " IE Escuela Normal Superior del MM" hizo la misma
aclaracion y dio una negativa a participar en el proceso, ``IE La
piedad'' hizo una negativa ante la posibilidad de de interaccion con sus
estudiantes sumado a que; ninguno de estos dos colegios poseia un
registro digital de sus estudiantes y se hizo una negativa a acceder a
dichos registros de manera fisica.

Instituciones como ``Maria auxiliadora de Bello'' o ``Institucion
Educativa Alfonso Lopez Pumarejo'' tambien del municipio de Bello,
dieron negativas a la toma de muestras e incluso mencionaron solo
estabar dispuestos a suministrar un listado de maximo 20 estudiantes.
Dichos datos no darian un analisis optimo para la toma de la muestra y
por tanto, la cantidad de colegios rechazados para el estudio fueron 9
en total.

El COLEGIO CAMPESTRE MONTAIGNE dio paso a el estudio con las
reestricciones ya mencionadas y el muestreo ha sido realizado con el
reducido abanico de posibilidades que este representa. Con esto
presente, se entiende que el trabajo ha sido limitado por factores
externos y la constante alternancia y PANDEMIA tambien imposibilitó en
gran medida el analisis para una correcta toma de datos para los
investigadores.

Finalmente, se aclara que el colegio en el cual se realiza la
investigacion; otorga dos (2) bases de datos para trabajar con ella. La
del 2019 y 2020. En esta base de datos se tiene nombre de los padres,
acudiente, ID, promedio academico obtenido, total de asignaturas
perdidas y ganadas.

\textbf{Parámetros a estimar}

\begin{itemize}
\tightlist
\item
  \(\mu_i\)=La media muestral del promedio obtenido al finalizar el año
  escolar (2019-2020) de los estudiantes del colegio Campestre
  Montaigne.
\item
  \(P_i\)=Proporción de estudiantes que ha aprobado o reprobado
  asignaturas al finalizar el año escolar en el colegio Campestre
  Montaigne.
\item
  \(A_i\)=El total de estudiantes que perdieron una asignatura al
  finalizar el año escolar (2019-2020) del colegio Campestre Montaigne.
\end{itemize}

\textbf{Población objetivo}

Todos los estudiantes que estan matriculados en el colegio Campestre
Montaigne durante los años 2019-2020.

\textbf{Marco muestral}

Listados de todos los estudiantes que estan matriculados en el colegio
Campestre Montaigne durante los años 2019-2020.

\textbf{Elemento de muestreo}

Cada uno de los estudiantes que estan matriculados en el colegio
Campestre Montaigne durante los años 2019-2020.

Para responder la pregunta problema (¿Como ha afectado la virtualidad
dada en la pandemia por el COVID-19 en la institucion educativa COLEGIO
CAMPESTRE MONTAIGNE?) se decide realizar un analisis de las notas
obtenidas por los estudiantes en el año inmediatamente anterior y el año
de la pandemia.

Se desea analizar si en efecto, la virtualidad afectó de forma positiva
o negativa el rendimiento academico de los estudiantes; entendiendo que
dicho rendimiento academico ha de ser medido en terminos del promedio
academico del estudiante. Es decir, un promedio academico alto (en
escala de 1 a 5) representa un buen rendimiento vs un promedio academico
bajo, representa un peor rendimiento.

Se entiende, dado esto; que si un alumno en el año 2019 tenia un
promedio de 4.8 y en el 2020 dicho promedio descendio a 4, se diria que
su rendimiento academico se vio afectado de manera significativa.

Si un estudiante tiene un promedio academico en el año 2019 de 3.5 y en
el año 2020 su siguiente promedio fue de 4.5 se identifica un aumento
considerable en su promedio y se asume que la vitrualidad lo ha
beneficiado.

Si, finalmente, un estudainte tenia un promedio de 3.5 en el año 2019 y
en el año siguiente su promedio registrado es de 3.6 o ya sea de 3.4 se
dice que dicho promedio se mantiene vigente ya que es natural que el
promedio academico oscile alrededor de un valor en circunstancias
normales.

Si este es el ultimo escenario, se dice que el estudiante no se vio
afectado de ninguna manera en sus estudios, para bien o para mal; de la
virtualidad y la pandemia actual.

\textbf{Tipo de muestreo}

La población tiene un total de 80 estudiantes,estos estan divididos en
grados los cuales al ser heterogeneos entre sí, toman el nombre de
estratos,se encuentran distribuidos en 4 estratos; el grado octavo,
noveno, decimo y once.

Para este caso y debido a la ausencia de muchos datos, se decide
trabajar \emph{NO} por grado uno a uno (es decir comparar los promedios
del grado 11 del 2019 vs los del grado 11 del 2020) puesto a que,
nuevamente, no se tienene algunos datos para realizar dicho analisis
(grado 11 del 2019 no se tiene informacion).

Por tal motivo, se decide comparar el rendimiento de \emph{Un mismo
grupo de estudiantes}. Es decir; se toma a los estudiantes de grado
decimo en 2019 y a los estudiantes de grado 11 del 2020 (Que son el
mismo grupo de estudio pero al año siguiente). De esta manera, se hace
el analisis sobre como afectó la virtualidad a los estudiantes del
colegio seleccionado (En la totalidad de los datos obtenidos) pero
discriminandolos de tal manera que se vea uno a uno dicho analisis.

\textbf{Muestreo Aleatorio Estratificado (MAE):}

El objetivo del MAE es maximizar la información obtenida teniendo en
cuenta que la variabilidad de la población no es homogénea. Inicialmente
se plantea realizar un muestreo para cada estrato, se opta por
seleccionar de manera aleatoria ciertos estudiantes de la base de datos
del colegio.

Algunas ventajas de usar MAE son:

\begin{itemize}
\tightlist
\item
  Produce un B (limite del error) mas pequeño con el mismo tamaño de
  muestra que MAS (Muestreo Aleatorio Simple); suponiendo que las
  mediciones de los estratos son homogéneas
\item
  Se reduce en tiempo y costos las observaciones. En este caso, la toma
  de datos y entrevistas se hubiesen facilitado.
\item
  La estimación de parámetros está dada por subgrupos, dando información
  mas especifica acerca de N.
\end{itemize}

Así entonces, se consideran a los cuatro estratos ya mencionados
(octavo, noveno, decimo y once) como subpoblaciones del análisis del
muestreo. Se realizarán estimación de los parámetros de interés en
dichas subpoblaciones en primera instancia y luego, dichas estimaciones
se utilizarán para estimar el parámetro poblacional de interés.

Se determina los tamaños de la población para los estratos mencionados
respectivamente como:

\begin{itemize}
\tightlist
\item
  \(K_{ij}\)= Número de estudiantes matriculados en el\\
\item
  j-esimo grado \(N_i\)=Poblacion en el Colegio, \(N=\sum(K_{ij})\)
\end{itemize}

Para estimar la varianza del parámetro de interés se realiza una muestra
piloto en la cual se selecciona una muestra aleatoria simple (M.A.S)
correspondiente al 35\% del total de la población de cada estrato. (Para
este caso en especifico al tener la lista de la población se elige por
facilidad el 35\%, aunque en casos con un N más grande este porcentaje
tiende a ser menor)

NOTA: En el archivo Excel adjunto se evidencian los datos suministrados
y se puede comparar las estimaciones realizadas para verificar y
constatar la veracidad de los próximos cálculos.

\begin{center}
\textbf{Elaboración del Muestreo}
\end{center}

\textbf{Resultados de la Muestra piloto año 2019}

\newpage

\textbf{Resultados de la Muestra piloto año 2020}

\textbf{Estimación del tamaño de la muestra para} \(\mu\)

Para este caso se va utilizar un N=80 que es la cantidad de estudiantes
matriculados en los años 2019 y 2020 con \(n_i\)=20, para i=1,2,3,4

\textbf{Para el 2019}

Sea: Estrato(1)Grado septimo,Estrato(2)Grado octavo,Estrato(3)Grado
noveno,Estrato(4)Grado décimo

De la muestra piloto se obtuvieron las siguientes varianzas:

\begin{itemize}
\item
  \(S^2_1\)=0.10402381
\item
  \(S^2_2\)=0.422380952
\item
  \(S^2_3\)=0.099047619
\item
  \(S^2_4\)=0.136190476
\end{itemize}

La formula tamaño de la muestra para la media con LEE B es:

\[n=\frac{\sum \frac{N_h^2*\sigma_h^2}{w_h}}{N^2*D+\sum^H_{h=1} N_h*\sigma_h^2}\]

Donde se tiene que \(w_h=\frac{n_h}{n}\) y ademas

Donde; \(D=Var[\bar{y_{est}}]=\frac{B^2}{Z^2}\) y
\(B=Z*\sqrt{Var[\bar{y_{est}}]}=Z*E.E(\bar{y}_{est})\)

Realizando las estimaciones pertinentes mediante la asignación de
Neyman, pues los costos se asumen iguales, se obtuvo un tamaño de
muestra de n=14 (afijacion de Neyman esta dada por
\((N_h*S_h)/\sum_{i=1}^H N_h*S_h\) )

Luego la muestra que se tomo para cada grado fue:

\(n_1\)=3,\(n_1\)=6,\(n_1\)=3,\(n_1\)=4

\newpage

\textbf{Para el 2020}

Sea: Estrato(1)Grado septimo,Estrato(2)Grado octavo,Estrato(3)Grado
noveno,Estrato(4)Grado décimo

De la muestra piloto se obtuvieron las siguientes varianzas:

\begin{itemize}
\item
  \(S^2_1\)=0.03032381
\item
  \(S^2_2\)=0.291428571
\item
  \(S^2_3\)=0.392380952
\item
  \(S^2_4\)=0.08952381
\end{itemize}

Realizando las estimaciones pertinentes mediante la asignación de
Neyman, pues los costos se asumen iguales, se obtuvo un tamaño de
muestra de n=15. (Usando la misma formula)

Luego la muestra que se tomo para cada grado fue:

\(n_1\)=2,\(n_1\)=5,\(n_1\)=6,\(n_1\)=4

\textbf{Estimación del tamaño de la muestra para} \(p\)

\textbf{Para el 2019}

Realizando las estimaciones pertinentes mediante la asignación de
Neyman, pues los costos se asumen iguales, se obtuvo un tamaño de
muestra de n=18.

Luego la muestra que se tomo para cada grado fue:

\(n_1\)=8,\(n_1\)=7,\(n_1\)=1,\(n_1\)=4

\textbf{Para el 2020}

Realizando las estimaciones pertinentes mediante la asignación de
Neyman, pues los costos se asumen iguales, se obtuvo un tamaño de
muestra de n=17.

Luego la muestra que se tomo para cada grado fue:

\(n_1\)=7,\(n_1\)=4,\(n_1\)=6,\(n_1\)=1

\newpage
\begin{center}
\textbf{Muestreo Definitivo}
\end{center}

Se enumeraron en orden de lista del 1 al 20 los estudiantes matriculados
en los estratos 1,2,3,4, Luego con la función sample de R se
seleccionaron aleatoriamente los estudiantes según el número de la
muestra correspondiente a cada estrato.

\textbf{Individuos seleccionados 2019}\(\mu\)

\begin{itemize}
\tightlist
\item
  Estrato 1: Grado septimo 7 6 2
\item
  Estrato 2: Grado octavo 14 11 1 18 4 8
\item
  Estrato 3: Grado noveno 8 11 14
\item
  Estrato 4: Grado decimo 4 18 20 17
\end{itemize}

\textbf{Individuos seleccionados 2020}\(\mu\)

\begin{itemize}
\tightlist
\item
  Estrato 1: Grado octavo 6 17
\item
  Estrato 2: Grado noveno 5 14 3 4 10
\item
  Estrato 3: Grado decimo 5 20 19 15 9 12
\item
  Estrato 4: Grado once 3 19 14 7
\end{itemize}

\textbf{Individuos seleccionados 2019}\(p\)

\begin{itemize}
\tightlist
\item
  Estrato 1: Grado septimo
\item
  Estrato 2: Grado octavo
\item
  Estrato 3: Grado noveno 1
\item
  Estrato 4: Grado decimo 6 15 11 3
\end{itemize}

\textbf{Individuos seleccionados 2020}\(p\)

\begin{itemize}
\tightlist
\item
  Estrato 1: Grado octavo\\
\item
  Estrato 2: Grado noveno
\item
  Estrato 3: Grado decimo 17 8 5 6 10 16
\item
  Estrato 4: Grado once 17
\end{itemize}

\newpage
\begin{center}
\textbf{Estimación de los parámetros 2019}
\end{center}

Sea Y el promedio anual de un estudiante en el colegio Campestre
Montaigne (variable cuantitativa), y asociados a esta variable, sea
\(\mu\) el promedio academico anual por grado en el colegio Campestre
MOntaigne(media poblacional de Y)

La siguiente tabla transpone y explica los datos del enunciado:

\begin{longtable}[]{@{}lllll@{}}
\toprule
\(Estrato\) & \(N_i\) & \(n_i\) & \(\bar{y}_i\) &
\(s_i^2\)\tabularnewline
\midrule
\endhead
1 & 20 & 3 & 3.43333 & 0.20333\tabularnewline
2 & 20 & 6 & 3.933333 & 0.466666\tabularnewline
3 & 20 & 3 & 3.833333 & 0.083333\tabularnewline
4 & 20 & 4 & 4.45 & 0.096667\tabularnewline
\bottomrule
\end{longtable}

\begin{itemize}
\item
  \(L = 4\) Grados en el Colegio Campestre Montaigne (número de
  estratos).
\item
  \(N = \sum_{i = 1}^L N_i = \sum_{i = 1}^4 N_i = 20 + 20 + 20 + 20 = 80\)
  Número de estudiantes matriculados en el Colegio Campestre Montaigne
  (tamaño de la población).
\item
  \(n = \sum_{i = 1}^L n_i = \sum_{i = 1}^4 n_i = 3 + 6 + 3+ 4 = 16\)
  Estudiantes seleccionadas (tamaño total de la muestra) mediante una
  MAE.
\item
  \(Z_{\alpha/2}= 1.96\)
\end{itemize}

Sea \(\mu_i,\, (i=1,2,3,4)\) el promedio anual por estudiante \(i\) del
Colegio Campestre Montaigne Sabemos que, un IC del \((1 - \alpha)100\%\)
para \(\mu_i\) es:
\[\bar{y}_i \,\pm\, t_{\alpha/2}\, \text{se}(\bar{y}_i)\]

se necesitan los valores de los cuantiles \(t_{0.025}\) de la
distribución \(t\) y las estimaciones
\(\text{se}(\bar{y}_i) = \sqrt{\widehat{V}(\bar{y}_i)}\), donde
\(\widehat{V}(\bar{y}_i) = (\frac{N_i - n_i}{N_i})\frac{s_i^2}{n_i}\)

\begin{longtable}[]{@{}llll@{}}
\toprule
\(Estrato\) & \(\widehat{V}(\bar{y}_i)\) & \(\sqrt{Var(y_i)}\) &
\(B\)\tabularnewline
\midrule
\endhead
1 & \((\frac{20 - 3}{20})\frac{0.20333}{3}=0.0577611\) & 0.2400231 &
1.03272\tabularnewline
2 & \((\frac{20 - 6}{20})\frac{0.466666}{6}=0.054444\) & 0.23 &
0.46346104\tabularnewline
3 & \((\frac{20 - 3}{20})\frac{0.083333}{3}=0.02361111\) & 0.15365 &
0.660734\tabularnewline
4 & \((\frac{20 - 4}{20})\frac{0.096667}{4}=0.019333\) & 0.139043 &
0.386049\tabularnewline
\bottomrule
\end{longtable}

\begin{longtable}[]{@{}ll@{}}
\toprule
\(Estrato\) & IC del 95\% para \(\mu_=y_i\pm B\)\tabularnewline
\midrule
\endhead
1 & \(3.43333\pm1.03272\)={[}2.400,4.46603{]}\tabularnewline
2 & \(3.933333\pm0.46346104\)={[}3.469871,4.396794{]}\tabularnewline
3 & \(3.833333\pm0.660734\)={[}3.1725,4.49406{]}\tabularnewline
4 & \(4.45\pm0.386049\)={[}4.063951,4.836049{]}\tabularnewline
\bottomrule
\end{longtable}

Ahora con una confianza del 95\%

\begin{itemize}
\tightlist
\item
  Se estima que el promedio anual en el grado séptimo del Colegio
  Campestre Montaigne esta entre 2.400 y 4.46603
\item
  Se estima que el promedio anual en el grado octavo del Colegio
  Campestre Montaigne esta entre 3.469871 y 4.396794
\item
  Se estima que el promedio anual en el grado noveno del Colegio
  Campestre Montaigne esta entre 3.1725 y 4.49406
\item
  Se estima que el promedio anual en el grado decimo del Colegio
  Campestre Montaigne esta entre 4.063951 y 4.836049
\end{itemize}

Sabemos que:
\[\bar{y}_\text{st} = \frac{\sum_{i = 1}^L N_i\,\bar{y}_i}{N} = \frac{\sum_{i = 1}^4 N_i\,\bar{y}_i}{N} = \frac{20(3.43333 ) + 20(3.933333) + 20(3.833333)+ 20(4.45)}{80} = 3.9125\]

También, que un IC del \((1 - \alpha)100\%\) para \(\mu\) es:
\[\bar{y}_\text{st} \,\pm\, t_{\alpha/2, n - L}\, \text{se}(\bar{y}_\text{st})\]

Nuevamente, para una confianza del 95\% se tiene que \(\alpha = 0.05\),
por lo tanto el valor \(t_{0.025, 80 - 4}\) es el cuantil \(0.975\) de
una distribución \(t\) con 77 grados de libertad. Usando \(R\) se
obtiene \(t_{0.025, 77} = 1.991\).

Adicionalmente, se requiere la estimación de
\(\text{se}(\bar{y}_\text{st}) = \sqrt{\widehat{V}(\bar{y}_\text{st})}\),
donde \[
        \begin{aligned}
        \widehat{V}(\bar{y}_\text{st}) &= \widehat{V}\Big(\frac{\sum_{i = 1}^3 N_i\,\bar{y}_i}{N}\Big) = \frac{\sum_{i = 1}^3 N_i^2\,\widehat{V}(\bar{y}_i)}{N^2}\\[0.2cm]
        &= \frac{20^2( 0.0577611) + 20^2(0.054444) + 20^2(0.02361111) + 20^2( 0.019333)}{80^2} = 0.009696825
        \end{aligned}
        \]

por tanto,
\(\text{se}(\bar{y}_\text{st}) = \sqrt{ 0.009696825} = 0.0985\). Luego,
un IC del 95\% para \(\mu\) es:
\[3.9125 \,\pm\, 1.991(0.6698) = [3.7164; 4.1086]\]

Y así, con una confianza del 95\% se estima que el promedio anual por
grado en el colegio Campestre Montaigne en el año 2019 esta entre 3.7164
y 4.1086

Sea p la proporción de estudiantes en cada grado que pierden materias
(proporcion de los elementos con el atributo) y sea \(a_i\) el número de
estudiantes que pierden asignaturas (atributo) La siguiente tabla
transpone y explica los datos dados en el enunciado:

\begin{longtable}[]{@{}lllll@{}}
\toprule
\(Estrato\) & \(N_i\) & \(n_i\) & \(a_1\) & \(p_i\)\tabularnewline
\midrule
\endhead
1 & 20 & 8 & 7 & 3/8=0.375\tabularnewline
2 & 20 & 7 & 0 & 0/7=0\tabularnewline
3 & 20 & 1 & 0 & 0/1=0\tabularnewline
4 & 20 & 4 & 0 & 0/4=0\tabularnewline
\bottomrule
\end{longtable}

Luego, se necesitan las estimaciones
\(\text{se}(\widehat{p}_i) = \sqrt{\widehat{V}(\widehat{p}_i)}\), donde
\(\widehat{V}(\widehat{p}_i) = (\frac{N_i - n_i}{N_i})\frac{\widehat{p}_i(1-\widehat{p}_i)}{n_i - 1}\).

\begin{longtable}[]{@{}llll@{}}
\toprule
\(Estrato\) & \(\widehat{V}(\bar{p}_i)\) & \(\sqrt{Var(p_i)}\) &
\(B\)\tabularnewline
\midrule
\endhead
1 & \((\frac{20 - 8}{20})(\frac{0.375(1-0.375)}{7})=0.20008\) & 0.141736
& 0.334498\tabularnewline
2 & 0 & 0 & 0\tabularnewline
3 & 0 & 0 & 0\tabularnewline
4 & 0 & 0 & 0\tabularnewline
\bottomrule
\end{longtable}

\begin{longtable}[]{@{}ll@{}}
\toprule
\(Estrato\) & IC del 95\% para \(p_=p_i\pm B\)\tabularnewline
\midrule
\endhead
1 & \(0.375\pm00.334498\)={[}0.04050,0.41550{]}\tabularnewline
\bottomrule
\end{longtable}

Con una confianza del 95\% se estima que el porcentaje de estudiantes
que perdieron asignaturas en el grado septimo estuvo ente el 4\% y el
41\%, mientras que en los grados octavo,noveno y decimo fue casi nula.

Luego, sabemos que: \[
        \begin{aligned}
        \widehat{p}_\text{st} &= \frac{\sum_{i = 1}^L N_i\,\widehat{p}_i}{N} = \frac{\sum_{i = 1}^3 N_i\,\widehat{p}_i}{N}\\[0.2cm]
        &= \frac{20(0.375) + 20(0) + 20(0) + 20(0)}{80} = 0.09375
        \end{aligned}
        \]

También, que un IC del \((1 - \alpha)100\%\) para \(p\) es:
\[\widehat{p}_\text{st} \,\pm\, t_{\alpha/2, n - L}\, \text{se}(\widehat{p}_\text{st})\]

Nuevamente, para una confianza del 95\% se tiene que \(\alpha = 0.05\),
por lo tanto el valor \(t_{0.025, 80 - 4}\) es el cuantil \(0.975\) de
una distribución \(t\) con 76 grados de libertad. Usando \(R\) se tiene
que \(t_{0.025, 76} = 1.992\).

Adicionalmente, se requiere la estimación de
\(\text{se}(\widehat{p}_\text{st}) = \sqrt{\widehat{V}(\widehat{p}_\text{st})}\),
donde \[
        \begin{aligned}
        \widehat{V}(\widehat{p}_\text{st}) &= \widehat{V}\Big(\frac{\sum_{i = 1}^3 N_i\,\widehat{p}_i}{N}\Big) = \frac{\sum_{i = 1}^3 N_i^2\,\widehat{V}(\widehat{p}_i)}{N^2}\\[0.2cm]
        &= \frac{20^2( 0.20008) + 20^2(0) + 20^2(0) + 20^2(0)}{80^2} = 0.012505
        \end{aligned}
        \]

por tanto,
\(\text{se}(\widehat{p}_\text{st}) = \sqrt{0.012505} = 0.1118\). Luego,
un IC del 95\% para \(p\) es:
\[0.09375\,\pm\,  1.992(0.1118) = [-0.129; 0.3165].\]

\newpage
\begin{center}
\textbf{Estimación de los parámetros 2020}
\end{center}

Sea Y el promedio anual de un estudiante en el colegio Campestre
Montaigne (variable cuantitativa), y asociados a esta variable, sea
\(\mu\) el promedio academico anual por grado en el colegio Campestre
MOntaigne(media poblacional de Y)

La siguiente tabla transpone y explica los datos del enunciado:

\begin{longtable}[]{@{}lllll@{}}
\toprule
\(Estrato\) & \(N_i\) & \(n_i\) & \(\bar{y}_i\) &
\(s_i^2\)\tabularnewline
\midrule
\endhead
1 & 20 & 2 & 3.1 & 0.02\tabularnewline
2 & 20 & 5 & 3.38 & 0.137\tabularnewline
3 & 20 & 6 & 3.8 & 0.328\tabularnewline
4 & 20 & 4 & 4.125 & 0.2025\tabularnewline
\bottomrule
\end{longtable}

\begin{itemize}
\item
  \(L = 4\) Grados en el Colegio Campestre Montaigne (número de
  estratos).
\item
  \(N = \sum_{i = 1}^L N_i = \sum_{i = 1}^4 N_i = 20 + 20 + 20 + 20 = 80\)
  Número de estudiantes matriculados en el Colegio Campestre Montaigne
  (tamaño de la población).
\item
  \(n = \sum_{i = 1}^L n_i = \sum_{i = 1}^4 n_i = 2 + 5 + 6+ 4 = 17\)
  Estudiantes seleccionadas (tamaño total de la muestra) mediante una
  MAE.
\item
  \(Z_{\alpha/2}= 1.96\)
\end{itemize}

Sea \(\mu_i,\, (i=1,2,3,4)\) el promedio anual por estudiante \(i\) del
Colegio Campestre Montaigne Sabemos que, un IC del \((1 - \alpha)100\%\)
para \(\mu_i\) es:
\[\bar{y}_i \,\pm\, t_{\alpha/2}\, \text{se}(\bar{y}_i)\]

se necesitan los valores de los cuantiles \(t_{0.025}\) de la
distribución \(t\) y las estimaciones
\(\text{se}(\bar{y}_i) = \sqrt{\widehat{V}(\bar{y}_i)}\), donde
\(\widehat{V}(\bar{y}_i) = (\frac{N_i - n_i}{N_i})\frac{s_i^2}{n_i}\)

\begin{longtable}[]{@{}llll@{}}
\toprule
\(Estrato\) & \(\widehat{V}(\bar{y}_i)\) & \(\sqrt{Var(y_i)}\) &
\(B\)\tabularnewline
\midrule
\endhead
1 & \((\frac{20 - 2}{20})\frac{0.02}{2}=0.009\) & 0.09486833 &
1.03272\tabularnewline
2 & \((\frac{20 - 5}{20})\frac{0.137}{5}=0.0822\) & 0.29 &
0.61823563\tabularnewline
3 & \((\frac{20 - 6}{20})\frac{0.328}{6}=0.03826667\) & 0.195618 &
0.502739\tabularnewline
4 & \((\frac{20 - 4}{20})\frac{0.2025}{4}=0.0405\) & 0.201246 &
0.122446\tabularnewline
\bottomrule
\end{longtable}

\begin{longtable}[]{@{}ll@{}}
\toprule
\(Estrato\) & IC del 95\% para \(\mu_=y_i\pm B\)\tabularnewline
\midrule
\endhead
1 & \(3.1\pm1.03272\)={[}2.06,4.13272{]}\tabularnewline
2 & \(3.38\pm0.61823563\)={[}2.76176,3.99823{]}\tabularnewline
3 & \(3.833333\pm0.346645\)={[}3.29726,4.3027399{]}\tabularnewline
4 & \(4.125\pm0.122446\)={[}4.012554,4.237446{]}\tabularnewline
\bottomrule
\end{longtable}

Ahora con una confianza del 95\%

\begin{itemize}
\tightlist
\item
  Se estima que el promedio anual en el grado octavo del Colegio
  Campestre Montaigne esta entre
\item
  Se estima que el promedio anual en el grado noveno del Colegio
  Campestre Montaigne esta entre 2.871616 y 3.88838
\item
  Se estima que el promedio anual en el grado decimo del Colegio
  Campestre Montaigne esta entre 3.29726 y 4.3027399
\item
  Se estima que el promedio anual en el grado once del Colegio Campestre
  Montaigne esta entre 4.012554 y 4.237446
\end{itemize}

Sabemos que:
\[\bar{y}_\text{st} = \frac{\sum_{i = 1}^L N_i\,\bar{y}_i}{N} = \frac{\sum_{i = 1}^4 N_i\,\bar{y}_i}{N} = \frac{20(3.1 ) + 20(3.38) + 20(3.8)+ 20(4.125 )}{80} = 3.6013\]

También, que un IC del \((1 - \alpha)100\%\) para \(\mu\) es:
\[\bar{y}_\text{st} \,\pm\, t_{\alpha/2, n - L}\, \text{se}(\bar{y}_\text{st})\]

Nuevamente, para una confianza del 95\% se tiene que \(\alpha = 0.05\),
por lo tanto el valor \(t_{0.025, 80 - 4}\) es el cuantil \(0.975\) de
una distribución \(t\) con 77 grados de libertad. Usando \(R\) se
obtiene \(t_{0.025, 77} = 1.991\).

Adicionalmente, se requiere la estimación de
\(\text{se}(\bar{y}_\text{st}) = \sqrt{\widehat{V}(\bar{y}_\text{st})}\),
donde \[
        \begin{aligned}
        \widehat{V}(\bar{y}_\text{st}) &= \widehat{V}\Big(\frac{\sum_{i = 1}^3 N_i\,\bar{y}_i}{N}\Big) = \frac{\sum_{i = 1}^3 N_i^2\,\widehat{V}(\bar{y}_i)}{N^2}\\[0.2cm]
        &= \frac{20^2( 0.02) + 20^2(0.137) + 20^2(0.328) + 20^2(0.2025)}{80^2} = 0.04296875
        \end{aligned}
        \]

por tanto,
\(\text{se}(\bar{y}_\text{st}) = \sqrt{ 0.04296875} = 0.2073\). Luego,
un IC del 95\% para \(\mu\) es:
\[ 3.6013 \,\pm\, 1.991( 0.2073) = [3.1886; 4.014]\]

Y así, con una confianza del 95\% se estima que el promedio anual por
grado en el colegio Campestre Montaigne en el año 2020 esta entre 3.1886
y 4.014

Sea p la proporción de estudiantes en cada grado que pierden materias
(proporcion de los elementos con el atributo) y sea \(a_i\) el número de
estudiantes que pierden asignaturas (atributo) La siguiente tabla
transpone y explica los datos dados en el enunciado:

\begin{longtable}[]{@{}lllll@{}}
\toprule
\(Estrato\) & \(N_i\) & \(n_i\) & \(a_1\) & \(p_i\)\tabularnewline
\midrule
\endhead
1 & 20 & 7 & 5 & 4/7=0.571428\tabularnewline
2 & 20 & 4 & 1 & 1/4=0.25\tabularnewline
3 & 20 & 6 & 1 & 1/6=0.16666\tabularnewline
4 & 20 & 1 & 0 & 0/1=0\tabularnewline
\bottomrule
\end{longtable}

Luego, se necesitan las estimaciones
\(\text{se}(\widehat{p}_i) = \sqrt{\widehat{V}(\widehat{p}_i)}\), donde
\(\widehat{V}(\widehat{p}_i) = (\frac{N_i - n_i}{N_i})\frac{\widehat{p}_i(1-\widehat{p}_i)}{n_i - 1}\).

\begin{longtable}[]{@{}llll@{}}
\toprule
\begin{minipage}[b]{(\columnwidth - 3\tabcolsep) * \real{0.32}}\raggedright
\(Estrato\)\strut
\end{minipage} &
\begin{minipage}[b]{(\columnwidth - 3\tabcolsep) * \real{0.22}}\raggedright
\(\widehat{V}(\bar{p}_i)\)\strut
\end{minipage} &
\begin{minipage}[b]{(\columnwidth - 3\tabcolsep) * \real{0.15}}\raggedright
\(\sqrt{Var(p_i)}\)\strut
\end{minipage} &
\begin{minipage}[b]{(\columnwidth - 3\tabcolsep) * \real{0.30}}\raggedright
\(B\)\strut
\end{minipage}\tabularnewline
\midrule
\endhead
\begin{minipage}[t]{(\columnwidth - 3\tabcolsep) * \real{0.32}}\raggedright
1\strut
\end{minipage} &
\begin{minipage}[t]{(\columnwidth - 3\tabcolsep) * \real{0.22}}\raggedright
\((\frac{20 - 7}{20})(\frac{0.5714(1-0.5714)}{6})=0.026531\)\strut
\end{minipage} &
\begin{minipage}[t]{(\columnwidth - 3\tabcolsep) * \real{0.15}}\raggedright
0.162883\strut
\end{minipage} &
\begin{minipage}[t]{(\columnwidth - 3\tabcolsep) * \real{0.30}}\raggedright
0.3990664\strut
\end{minipage}\tabularnewline
\begin{minipage}[t]{(\columnwidth - 3\tabcolsep) * \real{0.32}}\raggedright
2\strut
\end{minipage} &
\begin{minipage}[t]{(\columnwidth - 3\tabcolsep) * \real{0.22}}\raggedright
\((\frac{20 - 4}{20})(\frac{0.25(1-0.25)}{3})=0.05\)\strut
\end{minipage} &
\begin{minipage}[t]{(\columnwidth - 3\tabcolsep) * \real{0.15}}\raggedright
0.2236\strut
\end{minipage} &
\begin{minipage}[t]{(\columnwidth - 3\tabcolsep) * \real{0.30}}\raggedright
0.7116\strut
\end{minipage}\tabularnewline
\begin{minipage}[t]{(\columnwidth - 3\tabcolsep) * \real{0.32}}\raggedright
3\strut
\end{minipage} &
\begin{minipage}[t]{(\columnwidth - 3\tabcolsep) * \real{0.22}}\raggedright
\((\frac{20 - 6}{20})(\frac{0.166(1-0.166)}{5})=0.019382\)\strut
\end{minipage} &
\begin{minipage}[t]{(\columnwidth - 3\tabcolsep) * \real{0.15}}\raggedright
0.13921\strut
\end{minipage} &
\begin{minipage}[t]{(\columnwidth - 3\tabcolsep) * \real{0.30}}\raggedright
0.357794\strut
\end{minipage}\tabularnewline
\begin{minipage}[t]{(\columnwidth - 3\tabcolsep) * \real{0.32}}\raggedright
4\strut
\end{minipage} &
\begin{minipage}[t]{(\columnwidth - 3\tabcolsep) * \real{0.22}}\raggedright
0\strut
\end{minipage} &
\begin{minipage}[t]{(\columnwidth - 3\tabcolsep) * \real{0.15}}\raggedright
0\strut
\end{minipage} &
\begin{minipage}[t]{(\columnwidth - 3\tabcolsep) * \real{0.30}}\raggedright
0\strut
\end{minipage}\tabularnewline
\bottomrule
\end{longtable}

\begin{longtable}[]{@{}ll@{}}
\toprule
\(Estrato\) & IC del 95\% para \$ p\_=\hat p\pm B\$\tabularnewline
\midrule
\endhead
1 & \(3.1\pm1.03272\)={[}2.06,4.13272{]}\tabularnewline
2 & \(3.38\pm0.61823563\)={[}2.76176,3.99823{]}\tabularnewline
3 & \(3.833333\pm0.346645\)={[}3.29726,4.3027399{]}\tabularnewline
4 & \(4.125\pm0.122446\)={[}4.012554,4.237446{]}\tabularnewline
\bottomrule
\end{longtable}

\hypertarget{conclusiones}{%
\section{Conclusiones:}\label{conclusiones}}

\emph{Sobre el promedio:}

Al analizar los datos obtenidos de las muestras podemos afirmar a un
nivel de confianza del 95\% que la nota promedio en el colegio MONTAIGNE
en el año 2019 fue de 3.9 y en el 2020 fue de 3.6

Como se menciono con anterioridad, al tener una diferencia de 0.3 en la
nota, se puede afirmar que la virtualidad SI afectó de manera negativa a
la poblacion estudiantil de dicha institucion.

En las tablas anterioires se puede observar que el estrato 4 (decimo
2019-once 2020) es el estrato con mejor promedio de notas academicas.
Esto se explica a que, dicha institucion otorga becas e intercambios
internacionales e internacionales a el mejor estudiante. Es de esperar
pues; que el rendimiento, independiente de la pandemia o no, sea
ligeramente superior en los ultimos grados pues la beca se hace del
mejor promedio del grado 10 y 11. Esta informacion fue suministrada por
los directivos de la insitucion al preguntar del posible motivo por el
cual los promedios podrian ser mas altos en dicho estrato.

\emph{Sobre la proporcion y el total poblacional}

Acerca de la proporcion se tienen cierta redundancia en el calculo de
dicha proporcion. La mejor aproximacion para la toma de datos no es un
metodo continuo (y esto queda evidenciado al ver los IC negativos). Se
recomienda usar un metodo discreto para el calculo de la proporcion
poblacional a traves de una aproximacion binomial pues los datos con los
que se cuentan son muy pocos. (tomando como referencia, la cantidad de
asignaturas perdidas)

Con esto en mente, al ver la cantidad de estudiantes que pierden
asignaturas en ciertos estratos es practicamente nula (Por ejemplo,
grado 10 y 11 del 2019 y 2020); de la misma manera para los otros
estratos.

Por esto se plantea mirar el comportamiento del estimador de la
proporcion para cada estrato y con esto se tiene que; en el 2019 la
proporcion de estudiantes que perdio almenos una asignatura es de
0.09375 (cerca del 9.4 \% de los estudiantes totales).

Si pensamos en el total de estudiantes que han perdido una asignatura en
ese año bastaria con calcular 0.09375*N donde N es el numero total de
estudiantes(80) lo que implicaria que cerca de 7 estudiantes
aproximadamente, han perdido una asignatura en el año 2019. (Nuevamente
estos datos si son verificados, nos percatamos que a la luz de los
datos, la cantidad de estudiantes que pierden asignatura es muy baja)
por tanto, se plantea estas estimaciones a travez de una aproximacion
binomial.

\end{document}
